\documentclass[12pt]{article}

\usepackage{longtable}
\usepackage{multicol}
\usepackage{sbc-template}
\usepackage{graphicx,url}
\usepackage[brazil]{babel}
\usepackage[latin1]{inputenc}
\usepackage{lscape}
\usepackage{geometry}
\usepackage{float}
\usepackage{algorithm2e}
\usepackage{multicol}
\usepackage{amsmath}
\usepackage{amsfonts}
\usepackage{amssymb}
\usepackage{makeidx}
\usepackage{graphicx}
\usepackage{lmodern}
\usepackage{enumerate}
\usepackage{latexsym}
\usepackage{longtable}
\usepackage[all]{xy}
\usepackage{float}
\usepackage{lscape}
\usepackage{mathrsfs}
\usepackage{fancyhdr}
\usepackage{boxedminipage}
\usepackage{enumitem}
\usepackage{hyperref}


\sloppy

\title{Classificando imagens dos Simpsons}

\author{Leonardo Pontes Baiser\inst{1}, Marco Cezar Moreira de Mattos\inst{1},\\
		Rômulo Manciola Meloca\inst{1}}

\address{DACOM -- Universidade Tecnológica Federal do Paraná (UTFPR)\\
  Caixa Postal 271 -- 87301-899 -- Campo Mourão -- PR -- Brazil
  \email{\{lpbaiser, marco.cmm,rmeloca\}@gmail.com}
}

\begin{document}
	
\maketitle

\begin{abstract}

	This report shows the process of developing a program whose goal is to apply on the basis of images the 	concepts of artificial intelligence seen in the classroom along with the concepts of literature, therefore discriminate five classes referring to characters in the TV series \ textit {Simpsons}, these classes extract features and starting these characteristics apply sorting algorithms.

	\end{abstract}
     
\begin{resumo} 
  Este relatório mostra o processo de desenvolvimento de um programa cujo objetivo é aplicar sobre uma base de imagens os conceitos de inteligência artificial vistos em sala de aula juntamente com os conceitos da literatura, para tanto discriminamos cinco classes referentes aos personagens do seriado de TV \textit{Simpsons}, nestas classes extraímos características e apartir destas características aplicamos algoritmos de classificação.	
\end{resumo}

\section{Introdução}\label{sec:introducao}

	O aprendizado humano se dá através de três formas.
	Para a computação, não há outra forma de adquirir conhecimento, no sentido de conhecer determinadas informações, senão a aprendizagem.

	Podemos ``ensinar" um computador através de três maneiras, à moda dos seres humanos. São elas: O aprendizado supervisionado, o aprendizado não-supervisionado e o aprendizado por reforço. Em qualquer uma dessas três modalidades não há o aprendizado por..
	A saber, o aprendizado não supevisionado é aquele em que o computador precisa extrair informações que julgue relevantes sem qualquer apoio ou regra informada; Obviamente este é o modelo de aprendizagem que mais se busca. O aprendizado por reforço também exige do computador extrair conhecimento de um amontoado de informações, encontrando conexões a respeito dos dados analisados, contando ainda com uma bonificação a cada resposta certa e uma reclamação a cada resposta incorreta; À moda dos que recebem gorjeta. Há ainda, o aprendizado não-supervisionado, que fornece ao computador um conjunto de treinamento para que este aprenda a classificar itens já sabendo da real resposta de cada um destes e só em seguida seja lançado ao mundo para predizer itens não conhecidos.

	Para este trabalho, tratamos desta última categoria de classificadores. Neste, foi dado um problema que demandava escolher características de cada instância e classificá-las. A seção que segue, encarrega-se de elucidar o problema.


\section{O Problema}\label{sec:problema}

	Foi dado dois conjuntos, um para treino e outro para teste. Cada conjunto continha dezenas de imagens, cada uma com um (em algum casos mais que um) personagem do desenho animado ``Os Simpson"; São eles: Bart, Lisa, Homer, MARGE, e MAGGIE. Deste modo, cada imagem pertencente a um dos conjuntos configurou-se como uma instância de classificação.

	Cabia ao escopo deste trabalho escolher características para serem extraídas destas instâncias, bem como escolher os classificadores de aprendizagem não-supervisionada para operarem sobre os conjuntos.

	Como exaustivamente discutido por ***, mais importa a boa escolha de características que o algoritmo de classificação empregado, pois uma vez que as características não delimitam bem as classes, não há comparação que permita distinguí-las. Sob esse aspecto, consistia nisto a maior dificuldade do trabalho: extrair características das imagens dos personagens que fossem relevantes para distinguí-los.

	Aos olhos humanos distinguir os personagens da famílias ``Simpsons" é muito fácil, todavia, obter características que descrevam Sendo os personagens muito semelhantes, em termos de cor, encontrou-se grande dificuldade em.


\section{Organização da Solução}\label{sec:solucao}


\subsection{Diagramação}\label{sec:diagramacao}





\section{Implementação}



\section{Considerações Finais}


Exemplo de referência CC01a ~\cite{CC01}.

https://github.com/cjlin1/libsvm
https://github.com/saebyn/java-decision-tree
https://xanadu1010.wordpress.com/scripts-e-programas/transformada-rapida-de-fourier-com-java/

\bibliography{references}{}
\bibliographystyle{plain}

\end{document}	