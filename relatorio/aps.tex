\documentclass[12pt]{article}

\usepackage{longtable}
\usepackage{multicol}
\usepackage{sbc-template}
\usepackage{graphicx,url}
\usepackage[brazil]{babel}
\usepackage[latin1]{inputenc}
\usepackage{lscape}
\usepackage{geometry}
\usepackage{float}
\usepackage{algorithm2e}
\usepackage{multicol}
\usepackage{amsmath}
\usepackage{amsfonts}
\usepackage{amssymb}
\usepackage{makeidx}
\usepackage{graphicx}
\usepackage{lmodern}
\usepackage{enumerate}
\usepackage{latexsym}
\usepackage{longtable}
\usepackage[all]{xy}
\usepackage{float}
\usepackage{lscape}
\usepackage{mathrsfs}
\usepackage{fancyhdr}
\usepackage{boxedminipage}
\usepackage{enumitem}
\usepackage{hyperref}


\sloppy

\title{Classificando imagens dos Simpsons}

\author{Leonardo Pontes Baiser\inst{1}, Marco Cezar Moreira de Mattos\inst{1},\\
		Rômulo Manciola Meloca\inst{1}}

\address{DACOM -- Universidade Tecnológica Federal do Paraná (UTFPR)\\
  Caixa Postal 271 -- 87301-899 -- Campo Mourão -- PR -- Brazil
  \email{\{lpbaiser, marco.cmm,rmeloca\}@gmail.com}
}

\begin{document}
	
\maketitle

\begin{abstract}

	This report shows the process of developing a program whose goal is to apply on the basis of images the 	concepts of artificial intelligence seen in the classroom along with the concepts of literature, therefore discriminate five classes referring to characters in the TV series \ textit {Simpsons}, these classes extract features and starting these characteristics apply sorting algorithms.

	\end{abstract}
     
\begin{resumo} 
  Este relatório mostra o processo de desenvolvimento de um programa cujo objetivo é aplicar sobre uma base de imagens os conceitos de inteligência artificial vistos em sala de aula juntamente com os conceitos da literatura, para tanto discriminamos cinco classes referentes aos personagens do seriado de TV \textit{Simpsons}, nestas classes extraímos características e apartir destas características aplicamos algoritmos de classificação.	
\end{resumo}

\section{Introdução}\label{sec:introducao}

	Muito se discute, hoje em dia a utilização da inteligência artificial no mundo atual, o avanço 

\section{O Problema}\label{sec:problema}


\section{Organização da Solução}\label{sec:solucao}


\subsection{Diagramação}\label{sec:diagramacao}





\section{Implementação}

Dado as bibliotecas já implementadas, como mencionado na seção anterior, desenvolveu-se a solução com a linguagem de programação C.

Além das abstrações já citadas, para facilitar o desenvolvimento e torná-lo mais natural, tornando-o mais próximo ao conceito de objetos, utilizou-se apelidos característicos de objetos para estruturas já nomeadas (como por exemplo \emph{Connection} ao invés de \emph{connection\_t}) e definições de métodos redundantes (como por exemplo \emph{sendPackage()} em contraposição à \emph{CONN\_send()}). Tais fatores, permitiram o encapsulamento de algumas funcionalidades e possibilidade de fácil e rápida manutenção/refatoração do código.

Tendo sido mapeado os \emph{includes} no diagrama de classes, facilmente pode-se visualizar os módulos que os programas \emph{server} e \emph{client} deveriam incluir, mantendo a coerência com a distância que as partes (naturalmente) deveriam ter, com excessão da interface que os conecta, os pacotes definidos na biblioteca \emph{package.h} (que já conta com a inclusão da biblioteca \emph{connection.h}).

Colaborou-se o código com o auxílio do controle de versões git, onde o integrante Rômulo responsabilizou-se pela implementação do arquivo \emph{server.c} e das duas \emph{threads} \emph{requestHandler.c} e \emph{worker.c}. O núcleo da aplicação foi desenvolvido de maneira conjunta e interativa em relação as partes. O integrante Marco responsabilizou-se por todo o arquivo \emph{client.c} além do \textit{upload} do arquivo em \textit{multi-parts}.

\section{Considerações Finais}

Considera-se, por fim, que o desenvolvimento de um projeto que conta com várias \textit{threads}, o conceito de produtor-consumidor, um \textit{buffer} compartilhado onde apresentam-se condições de corrida bem específicas, a integração de vários módulos além da comunicação inter-processos via \textit{socket}, permite no mínimo alargar os conhecimentos e fixar o aprendizado de todos esses conceitos vastamente utilizados nas mais diversas aplicações atuais. Nesse espectro, salienta-se a importância de tal desenvolvimento e, sobretudo, a fase de projeto, que tanto agrega para a visualização panorâmica deste ponto em específico da disciplina de Sistemas Operacionais.

Conclui-se que aplicações de determinada escala demandam a produção de vários artefatos, que somente são produzidos após o fiel debruçar-se nas ideias e adiantar-se a respeito de todos os problemas que são solucionados e gerados a partir delas. Toma-se como proveito o pensar em soluções que possam ser integradas a demais programas e o pensar em abstrações e interfaces que possam ser escaladas e reutilizadas.

Exemplo de referência CC01a ~\cite{CC01}.

\bibliography{references}{}
\bibliographystyle{plain}

\end{document}	