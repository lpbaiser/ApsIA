\documentclass[12pt]{article}

\usepackage{sbc-template}
\usepackage{graphicx,url}
\usepackage[brazil]{babel}
\usepackage[latin1]{inputenc}
\usepackage{lscape}
\usepackage{geometry}
\usepackage{float}

\sloppy

\title{Algoritmo A* para resolução da Árvore Geradora de Rótulos Mínimos}

\author{Marco Cezar Moreira de Mattos\inst{1}, Rômulo Manciola Meloca\inst{1}}

\address{DACOM -- Universidade Tecnológica Federal do Paraná (UTFPR)\\
  Caixa Postal 271 -- 87301-899 -- Campo Mourão -- PR -- Brazil
  \email{\{marco.cmm,rmeloca\}@gmail.com}
}

\begin{document}

\maketitle

\begin{abstract}
  This report shows the development process of a ******, since the conception, showing the designed solution until its implementation phase. This report shows project decisions and outlined obstacles.
\end{abstract}
     
\begin{resumo} 
  Este relatório apresenta o emprego do algoritmo A* para a resolução da Árvore Geradora de Rótulos Mínimos, desenvolvimento de um software servidor de arquivos, desde a sua concepção, apresentando a solução projetada até a fase de implementação do mesmo. Lê-se neste decisões de projeto e obstáculos contornados.
\end{resumo}

\section{O Problema}\label{sec:problema}

Disponibilizado no \textit{moodle}, serviço de apoio educacional, chegou-se aos alunos o problema de transformar uma aplicação do modelo cliente-servidor \textit{single-thread} em um programa que permitisse o acesso de vários clientes, bem como que cada um deles pudesse requisitar uma listagem do diretório e/ou o \textit{download} de algum arquivo presente nele, tornando-o um servidor de arquivos. Além de permitir uma \textit{thread} para cada cliente, deveria-se utilizar o paradigma produtor-consumidor \emph{n:m}, onde várias \textit{threads} produziriam requisições em um buffer compartilhado e várias \textit{threads} consumidoras tratariam devidamente cada requisição.

Juntamente com o exemplo, foi disponibilizado também uma bliblioteca de funcionalidades que encapsulam as especificidades do protocolo TCP/IP com a linguagem de programação C, da qual o próprio exemplo dado a utilizava.

\section{Organização da Solução}\label{sec:solucao}

De início, fazia-se necessário a diagramação do problema para dar início ao desenvolvimento da solução, pois havia a necessidade de, em primeira instância, compreender o problema e projetar a solução, além de estabelecer uma linguagem comum para os desenvolvedores. Nesse sentido, utilizou-se o diagrama de classes e alguns outros esquemas de autoria própria.

Uma vez fornecida a biblioteca de encapsulamento do TCP/IP, já preparada para o desenvolvimento da solução, optou-se por implementar a solução na linguagem de programação C.

\subsection{Diagramação}\label{sec:diagramacao}

Como o diagrama de classes é uma ferramenta de modelagem de aplicações orientadas a objetos, foi preciso fazer algumas adaptações para o desenvolvimento procedural:
As classes com inicial maiúscula são estruturas, e seus elementos são os atributos;
As classes com inicial minúscula são arquivos;
Os atributos oriundos de classes-arquivos são variáveis globais;
Os modificadores de acesso são todos públicos, uma vez que não há nenhuma proteção nas variáveis (embora tenha-se optado por fazer o acesso apenas através de ``getters" e ``setters"); Os métodos de nome create* simulam o método construtor;
As associações representam os includes entre os arquivos;
Tipos primitivos são mantidos e ponteiros tratados como objetos (salvo tipos enumeráveis, que também são tratados como objetos), como por exemplo String ao invés de char*;
Associações bi-direcionais, mantêm a semântica de includes bi-direcionais.

Utilizou-se no diagrama cor cinza para os elementos previamente implementados e reutilizados, tons azuis para objetos inerentes ao processamento do servidor e tom amarelo para o cliente. Azul ciano para as interfaces de abstração dos dados a serem trafegados. Azul turquesa para as threads. Azul marinho para o servidor.

Segue a imagem do diagrama construído com o auxílio do \textit{software} gratuito de modelagem Astah.

\newgeometry{left=0cm,bottom=0cm,right=0cm,top=0cm}
\begin{landscape}
\centering
\begin{figure}[p]
\includegraphics[width=1.4\textwidth]{ClassDiagram.png}
\caption{Diagrama de Classes}
\label{fig:classDiagram}
\end{figure}
\end{landscape}
\restoregeometry

Acima visualizou-se o diagrama de classes mostrando os arquivos e estruturas criadas, bem como os métodos e atributos que detectou-se em cada objeto. Na subseção seguinte são detalhadas as interfaces criadas.

\subsection{Interfaces}

Lançou-se mão das bibliotecas já implementadas \emph{lista.h}, que abstraiu os detalhes de gerenciamento de uma lista encadeada e do módulo \emph{connection.h}, que abstraiu as especificidades do TCP/IP. Com o uso das bibliotecas mencionadas permitiu-se ainda que esses módulos sejam facilmente substituídos por outros, como por exemplo a troca de uma lista encadeada para uma lista de prioridades.

Trabalhando acima das camadas de \textit{software} mencionadas, ainda eram necessárias mais abstrações com a finalidade de permitir as \textit{threads} produtoras e consumidoras pudessem estabelecer um diálogo. Nesse sentido, criou-se a estrutura \textit{Request}, que congrega a conexão com o cliente (para permitir a \textit{worker-thread} responder adequadamente ao cliente), o tipo da requisição (Boas-vindas, encerramento de conexão, listagem de arquivos, \textit{download}, tamanho máximo suportado, ou outro), o caminho (para listagem de arquivos e pedido de \textit{download}), um \textit{status} para monitorar o andamento da requisição, e ainda, o valor do tamanho máximo de dados que o cliente suporta.

Uma outra decisão de projeto tomada pela equipe, foi a de criar ainda uma abstração a mais para o encapsulamento da mensagem que transitaria entre o cliente e o servidor. Assim, criou-se a estrutura \textit{Package} que conta com aquele mesmo tipo enumerável da requisição, os dados contidos no pacote, um tamanho total da mensagem a ser enviada bem como seu \textit{offset}, a fim de calcular, para cada pacote, qual é a sua porção na mensagem total, além de permitir o cálculo de quantos são os pacotes no total, para cada requisição do cliente.

Abaixo, segue um dos artefatos produzidos para representar o fluxo de dados durante o atendimento de uma requisição.

\newpage

Observa-se no fluxo de dados que a mensagem é encapsulada em pacotes (para além dos segmentos, pacotes e quadros do TCP/IP), de modo que ambos cliente e servidor precisam compreender essa linguagem. Uma vez que o pacote é recebido pelo servidor na sua \textit{thread} \textit{Request Handler}, responsável pelo cliente em questão, o pacote é interpretado e transformado em uma requisição, que é posta no \textit{buffer} compartilhado entre todas as \textit{threads}. Assim que disponível, alguma das \textit{Worker-threads} por sua vez, toma a requisição, a executa e responde para o cliente em forma de pacotes.

Avistava-se ainda um último problema de projeto, pois algumas requisições do cliente poderiam demandar várias respostas seguidas do servidor. A subseção seguinte discute esse ponto.

\subsection{Protocolo}

É verdade que utilizou-se o protocolo TCP/IP para mediar a transmissão de \textit{bytes} entre cliente e servidor e é verdade que este divide os dados em partes para enviá-lo, conforme o \textit{Maximum Transmission Unit} (MTU) da rede e ainda remonta os pacotes na ordem correta. Contudo, algumas requisições eventualmente precisariam de várias respostas ou por ser um arquivo demasiado grande ou porque o cliente suporta apenas pacotes menores.

Em qualquer dos casos, precisava-se criar um protocolo de comunicação que suportasse o envio de informações \textit{multi-part}, seja para \textit{download} de arquivos, ou por requisições do tipo \emph{ls} (listagem de arquivos) em diretórios muito extensos, que, sobretudo, fosse capaz de remontar os pacotes a exemplo do TCP/IP.

O problema residia no momento de perguntar ao cliente qual o tamanho máximo de dados que ele suportava, dado que a \textit{worker-thread} não tinha condições de escutar o cliente, uma vez que a \textit{thread} \textit{Request Handler} mantinha-se em escuta ao cliente. Também não seria possível que a \textit{thread} \textit{Request Handler} enviasse a resposta da pergunta para a \textit{worker thread} correta. Em todos os casos, fugia da responsabilidade da \textit{worker} escutar a conexão e fugia da responsabilidade do \textit{Request Handler} enviar pacotes.

Frente a essa dificuldade, pensou-se e criou-se o protocolo que pode ser visto na imagem a seguir.

A solução consiste no \textit{Request Handler} ``segurar"" os pacote de pedidos do tipo \emph{ls} e \emph{wget} e, ao invés de transformá-lo em uma requisição e colocá-lo no \textit{buffer} compartilhado, enviar para a fila de trabalho das \textit{worker-threads} uma requisição do tipo MAXPACKAGESIZE, que questiona ao cliente qual o tamanho máximo que ele suporta. Ao receber o pacote de resposta, a \textit{thread} \textit{Request Handler} acorda e pode prosseguir com o tratamento da requisição anterior, informando à \textit{worker-thread} qual o tamanho máximo de cada pacote.

Tendo sido preenchidas as maiores lacunas do problema e fechadas todas as arestas no tocante à dificuldades com a estruturação da solução, já era possível implementar a solução. A seção seguinte expressa essa etapa.

\section{Implementação}

Dado as bibliotecas já implementadas, como mencionado na seção anterior, desenvolveu-se a solução com a linguagem de programação C.

Além das abstrações já citadas, para facilitar o desenvolvimento e torná-lo mais natural, tornando-o mais próximo ao conceito de objetos, utilizou-se apelidos característicos de objetos para estruturas já nomeadas (como por exemplo \emph{Connection} ao invés de \emph{connection\_t}) e definições de métodos redundantes (como por exemplo \emph{sendPackage()} em contraposição à \emph{CONN\_send()}). Tais fatores, permitiram o encapsulamento de algumas funcionalidades e possibilidade de fácil e rápida manutenção/refatoração do código.

Tendo sido mapeado os \emph{includes} no diagrama de classes, facilmente pode-se visualizar os módulos que os programas \emph{server} e \emph{client} deveriam incluir, mantendo a coerência com a distância que as partes (naturalmente) deveriam ter, com excessão da interface que os conecta, os pacotes definidos na biblioteca \emph{package.h} (que já conta com a inclusão da biblioteca \emph{connection.h}).

Colaborou-se o código com o auxílio do controle de versões git, onde o integrante Rômulo responsabilizou-se pela implementação do arquivo \emph{server.c} e das duas \emph{threads} \emph{requestHandler.c} e \emph{worker.c}. O núcleo da aplicação foi desenvolvido de maneira conjunta e interativa em relação as partes. O integrante Marco responsabilizou-se por todo o arquivo \emph{client.c} além do \textit{upload} do arquivo em \textit{multi-parts}.

\section{Considerações Finais}

Considera-se, por fim, que o desenvolvimento de um projeto que conta com várias \textit{threads}, o conceito de produtor-consumidor, um \textit{buffer} compartilhado onde apresentam-se condições de corrida bem específicas, a integração de vários módulos além da comunicação inter-processos via \textit{socket}, permite no mínimo alargar os conhecimentos e fixar o aprendizado de todos esses conceitos vastamente utilizados nas mais diversas aplicações atuais. Nesse espectro, salienta-se a importância de tal desenvolvimento e, sobretudo, a fase de projeto, que tanto agrega para a visualização panorâmica deste ponto em específico da disciplina de Sistemas Operacionais.

Conclui-se que aplicações de determinada escala demandam a produção de vários artefatos, que somente são produzidos após o fiel debruçar-se nas ideias e adiantar-se a respeito de todos os problemas que são solucionados e gerados a partir delas. Toma-se como proveito o pensar em soluções que possam ser integradas a demais programas e o pensar em abstrações e interfaces que possam ser escaladas e reutilizadas.

\end{document}